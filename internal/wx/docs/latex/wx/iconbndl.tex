\section{\class{wxIconBundle}}\label{wxiconbundle}

This class contains multiple copies of an icon in different sizes,
see also \helpref{wxDialog::SetIcons}{wxdialogseticons} and
\helpref{wxTopLevelWindow::SetIcons}{wxtoplevelwindowseticons}.

\wxheading{Derived from}

No base class

\latexignore{\rtfignore{\wxheading{Members}}}

\membersection{wxIconBundle::wxIconBundle}\label{wxiconbundlewxiconbundle}

\func{}{wxIconBundle}{\void}

Default constructor.

\func{}{wxIconBundle}{\param{const wxString\& }{file}, \param{long }{type}}

Initializes the bundle with the icon(s) found in the file.

\func{}{wxIconBundle}{\param{const wxIcon\& }{icon}}

Initializes the bundle with a single icon.

\func{}{wxIconBundle}{\param{const wxIconBundle\& }{ic}}

Copy constructor.

\membersection{wxIconBundle::\destruct{wxIconBundle}}\label{wxiconbundledtor}

\func{}{\destruct{wxIconBundle}}{\void}

Destructor.

\membersection{wxIconBundle::AddIcon}\label{wxiconbundleaddicon}

\func{void}{AddIcon}{\param{const wxString\& }{file}, \param{long }{type}}

Adds all the icons contained in the file to the bundle;
if the collection already contains icons with the same
width and height, they are replaced by the new ones.

\func{void}{AddIcon}{\param{const wxIcon\& }{icon}}

Adds the icon to the collection; if the collection already
contains an icon with the same width and height, it is
replaced by the new one.

\membersection{wxIconBundle::GetIcon}\label{wxiconbundlegeticon}

\constfunc{const wxIcon\&}{GetIcon}{\param{const wxSize\& }{size}}

Returns the icon with the given size; if no such icon exists,
returns the icon with size wxSYS\_ICON\_X/wxSYS\_ICON\_Y;
if no such icon exists,
returns the first icon in the bundle. If size = wxSize( -1, -1 ),
returns the icon with size wxSYS\_ICON\_X/wxSYS\_ICON\_Y.

\constfunc{const wxIcon\&}{GetIcon}{\param{wxCoord }{size = -1}}

Same as GetIcon( wxSize( size, size ) ).

\membersection{wxIconBundle::operator=}\label{wxiconbundleoperatorassign}

\func{const wxIconBundle\&}{operator=}{\param{const wxIconBundle\& }{ic}}

Assignment operator.

